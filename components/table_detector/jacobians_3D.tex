%! Author = pbustos
%! Date = 22/02/25


% Preamble
\documentclass[11pt]{article}

% Packages
\usepackage{amsmath, amsfonts}
\usepackage[a4paper, total={6.5in, 8in}]{geometry}
\usepackage{amsfonts}
\setlength{\parindent}{0pt}
\setlength{\parskip}{1em} % Adjust the length as needed

% Document
\begin{document}

\title{Loss function: distance to the side of an orthohedron (object)}
\maketitle

    In our 3D object detection problem, the cuboid object is modeled by a 9-dimensional parameter vector
    \begin{equation}
        \mathbf{v} = \begin{bmatrix} x \\ y \\ z \\ \alpha \\ \beta \\ \gamma \\ \text{w} \\ \text{d} \\ \text{h} \end{bmatrix},\label{eq:equation_1}
    \end{equation}

    where $x,y,z$ define the position of the object in the room frame, $\alpha,\beta,\gamma$ define the rotation of the object in the room frame, and $w,d,h$ define the width, depth, and height of the table respectively.
    The objective is to define a differentiable distance from a point $p$ with coordinates in the object frame, to one side (top) of the object.

    Let the width of the orthohedron be \(\text{w}\), the depth \(\text{d}\) and the height \(\text{h}\).
    We define the transformation from the center of the object to the top side with an \(\text{Pose3}\) object:

    \begin{equation}
        T_{t2s} = \text{Pose3}\Bigl(\mathbf{R}=\mathbf{I},\; \mathbf{t} = \begin{bmatrix} 0 \\  0 \\ \frac{h}{2} \end{bmatrix}\Bigr),
    \end{equation}

    and the transformed point $ps$,

    \begin{equation}
        ps = \text{Pose3::transformTo}(p,H_{ps}(3x6), H_{pt}(3x3)) = \begin{bmatrix} x \\ y \\ z \end{bmatrix} \in \mathbb{R}^3,
    \end{equation}

    The method provides the Jacobians with respect to the pose parameters and the input point \(p\).

    We follow the same convention for all the sides of the orthohedron, where the normal vector points outwards in the positive $Z$ direction, the $X+$ direction is the width, and the $Y+$ direction is the depth.

    \section*{Distance computation}

    Let the top side of the object defined in the side coordinate system be given by the plane,
    \begin{equation}
        z = 0,
    \end{equation}
    with a rectangular boundary given by
    \begin{equation}
        \mathcal{R} = \{ (x,y,0) \mid -L_x \le x \le L_x,\; -L_y \le y \le L_y \}.
    \end{equation}

    where \(L_x = \frac{w}{2}\) and \(L_y = \frac{d}{2}\).

    The orthogonal projection of Point $ps$ onto the plane is

    \begin{equation}
        p_{\text{proj}} = \begin{bmatrix} ps_x \\ ps_y \\ 0 \end{bmatrix}.
    \end{equation}

    We now analyse the two possible cases for the projection of \(p\) onto the face of the object: falling inside and falling outside the rectangle \(\mathcal{R}\).

    \subsection*{Case 1: Projection Inside the Face}

        If the projection lies inside the rectangle, i.e.,
        \begin{equation}
            |ps_x| \le L_x \quad \text{and} \quad |ps_y| \le L_y,
        \end{equation}
        then the closest point on the face is \(p_{\text{proj}}\) itself and the (squared) distance from \(p\) to the face is simply the squared perpendicular distance:
        \begin{equation}
            D^2(p) = ps^2_z.
        \end{equation}

    \subsection*{Case 2: Projection Outside the Face}

        If the projection lies outside \(\mathcal{R}\), then the closest point \(q\) on the rectangle is the one that minimizes the Euclidean distance to \(p_{\text{proj}}\). Define the distances along the \(x\) and \(y\) directions by
        \begin{equation}
            d_x = \max\{0,\, |ps_x| - L_x\}, \qquad
            d_y = \max\{0,\, |ps_y| - L_y\}.
        \end{equation}
        Then, the squared distance from \(p\) to the face is given by
        \begin{equation}
            D^2(p) = z^2 + d_x^2 + d_y^2.
        \end{equation}

    \subsection*{Unified Formulation}

        One can combine the two cases into a single expression:
        \begin{equation}
            D^2(p) = z^2 + \left[ \max\{0,\, |ps_x| - L_x\} \right]^2 + \left[ \max\{0,\, |ps_y| - L_y\} \right]^2.
        \end{equation}

    \subsection*{Smooth Approximation}

        Since the \(\max\) function is not differentiable at zero, a smooth alternative is to use the softplus function to approximate the \(\max\) (or the ReLU). For example, define:
        \begin{equation}
            \text{softplus}(x) = \ln\left(1+\exp(x)\right),
        \end{equation}

        which approximates \(\max\{0,x\}\). Then, one can define smooth approximations of \(d_x\) and \(d_y\) by:

        \begin{equation}
            d_x \approx \text{softplus}(|x| - L_x), \qquad
            d_y \approx \text{softplus}(|y| - L_y),
        \end{equation}

        The derivative of the softplus function is given by:
        \begin{equation}
            \text{softplus}'(x) = \frac{\exp(x)}{1+\exp(x)} = \frac{1}{1+\exp(-x)}.
        \end{equation}

        For the absolute value, we can use a \textit{softabs()} function defined as,
        \begin{equation}
            \text{softabs}(x) = \sqrt{x^2 + \epsilon}, \qquad \text{where } \epsilon \text{ is a small constant}.
        \end{equation}

        The derivative of the softabs function is given by:
        \begin{equation}
            \text{softabs}'(x) = \frac{x}{\sqrt{x^2 + \epsilon}}.
        \end{equation}


        The smooth squared distance becomes,

        \begin{equation}
            \boxed{D(p,v) = \approx z^2 + \underbrace{\text{softplus}(softabs(x) - L_x)^2}_{dx^2} + \underbrace{\text{softplus}(softabs(y) - L_y)^2}_{dy^2}}
        \end{equation}




\section*{Computation of the Jacobian of the distance wrt to the parameter vector $v$}

        The derivative of the squared distance with respect to the parameter vector \(\mathbf{v}\) is given by:

        \begin{equation}
            \frac{\partial D}{\partial v} = \left[0,0,0,0,0,0, \frac{\partial D}{\partial w}, \frac{\partial D}{\partial d}, \frac{\partial D}{\partial h}\right]
        \end{equation}

        The derivatives with respect to the parameters $w,d,h$ are computed using the chain rule and the transformation matrix \(H_{pose}\).

        We define the intermediate variables:
        \begin{align}
            dx^2 &= \left[\text{softplus}(\text{softabs}(ps_x) - L_x)\right]^2,\\
            u^2 &= [softplus(v)]^2 \\
            s &= \text{softabs}(ps_{x}) = \sqrt{ps_{x}^2 + \epsilon}, \\
            v &= s - L_x, \\
            ps_{x} &= H_{point}(0,:)\, p, \\
            L_x &= \frac{width}{2} \\
        \end{align}

        \subsection*{Derivative of $D$ with respect to $w$}
            The derivative of \(D\) with respect to \(w\) is given by:
            \begin{equation}
                \frac{\partial D}{\partial w} = \frac{\partial D}{\partial dx^2} \cdot \frac{\partial dx^2}{\partial w}
            \end{equation}

            Since only the \(dx^2\) term depends on \(w\), we can compute the derivative of \(D\) with respect to \(dx^2\) first:

            \begin{equation}
                \frac{\partial D}{\partial w} = \frac{\partial dx^2}{\partial w}
            \end{equation}

            Using the chain rule, we can express the derivative of \(dx^2\) with respect to \(w\) as:
            \begin{equation}
                \frac{\partial dx^2}{\partial w} = \frac{\partial dx^2}{\partial u} \cdot \frac{\partial u}{\partial v} \cdot \frac{\partial v}{\partial s} \cdot \frac{\partial s}{\partial ps_{x}} \cdot \frac{\partial ps_{x}}{\partial p} \cdot \frac{\partial p}{\partial w}
            \end{equation}

            where
            \begin{align}
                \frac{\partial p}{\partial w} &= \begin{bmatrix} 1 \\ 0 \\ 0 \end{bmatrix} \\
                \frac{\partial ps_{x}}{\partial p_x} &= H_{point}(0,:) \\
                \frac{\partial s}{\partial ps_{x}} &= \frac{ps_{x}}{\sqrt{ps_{x}^2 + \epsilon}}\\
                \frac{\partial v}{\partial s} &= 1\\
                \frac{\partial u}{\partial v} &= \frac{1}{1 + e^{-(s - L_x)}}\\
                \frac{\partial dx^2}{\partial u} &= 2\,\text{softplus}(s - L_x)\\
            \end{align}

            Multiplying all the factors, we have:
            \begin{equation}
                \frac{\partial D}{\partial w} = \frac{\partial dx^2}{\partial w} = 2\,\text{softplus}(s - L_x) \cdot \frac{1}{1 + e^{-(s - L_x)}} \cdot \frac{ps_{x}}{\sqrt{ps_{x}^2 + \epsilon}} \cdot H_{point}(0,:) \cdot \begin{bmatrix} 1 \\ 0 \\ 0 \end{bmatrix}
            \end{equation}


    \subsection*{Derivative of $D$ with respect to $d$}
        The derivative of \(D\) with respect to \(d\) is given by:
        \begin{equation}
            \frac{\partial D}{\partial d} = \frac{\partial D}{\partial dy^2} \cdot \frac{\partial dy^2}{\partial d}
        \end{equation}

        Since only the \(dy^2\) term depends on \(d\), we can compute the derivative of \(D\) with respect to \(dy^2\) first:

        \begin{equation}
            \frac{\partial D}{\partial d} = \frac{\partial dy^2}{\partial d}
        \end{equation}

        Using the chain rule, we can express the derivative of \(dy^2\) with respect to \(d\) as:

        \begin{equation}
            \frac{\partial dy^2}{\partial d} = \frac{\partial dy^2}{\partial u_y} \cdot \frac{\partial u_y}{\partial v_y} \cdot \frac{\partial v_y}{\partial s_y} \cdot \frac{\partial s_y}{\partial ps_{y}} \cdot \frac{\partial ps_{y}}{\partial p_y} \cdot \frac{\partial p_y}{\partial d}
        \end{equation}

        where
        \begin{align}
            \frac{\partial p}{\partial d} &= \begin{bmatrix} 0 \\ 1 \\ 0 \end{bmatrix} \\
            \frac{\partial ps_{y}}{\partial p_y} &= H_{point}(1,:)\\
            \frac{\partial s_{y}}{\partial ps_{y}} &= \frac{ps_{y}}{\sqrt{ps_{y}^2 + \epsilon}}\\
            \frac{\partial v_{y}}{\partial s_{y}} &= 1\\
            \frac{\partial u_{y}}{\partial v_{y}} &= \frac{1}{1 + e^{-(s - L_y)}}\\
            \frac{\partial dy^2}{\partial u_y} &= 2\,\text{softplus}(s - L_y)\\
        \end{align}

        Multiplying all the factors, we have:
        \begin{equation}
            \frac{\partial D}{\partial d} = \frac{\partial D}{\partial dy^2}  = 2 \cdot
                                            \text{softplus}(s - L_y) \cdot
                                            \frac{1}{1 + e^{-(s - L_y)}} \cdot
                                            \frac{ps_{y}}{\sqrt{ps_{y}^2 + \epsilon}} \cdot
                                            H_{point}(1,:) \cdot
                                            \begin{bmatrix} 0 \\ 1 \\ 0 \end{bmatrix}
        \end{equation}

    \subsection*{Derivative of $D$ with respect to $h$}
        The derivative of \(D\) with respect to \(h\) is given by:
        \begin{equation}
            \frac{\partial D}{\partial h} = \frac{\partial D}{\partial ps_{z}} \cdot \frac{\partial ps_{z}}{\partial h}
        \end{equation}

        Since $D = ps^2_z$ plus two other terms that not depend on h,

        \begin{equation}
            \frac{\partial D}{\partial h} = \frac{\partial ps^2_{z}}{\partial ps_z} \cdot  \frac{\partial ps_z}{\partial z} \cdot \frac{\partial p_z}{\partial h} = 2 \cdot ps_z \cdot H_{point}(2,:) \cdot \begin{bmatrix} 0 \\ 0 \\ 1 \end{bmatrix}
        \end{equation}

    \subsubsection*{Final form of the Jacobian of D wrt v}
        The final Jacobian of the distance with respect to \(v\) is given by:

        \begin{equation}
            \frac{\partial D}{\partial v} = \left[0,0,0,0,0,0, \frac{\partial D}{\partial w}, \frac{\partial D}{\partial d}, \frac{\partial D}{\partial h}\right]
        \end{equation}

        where

        \begin{equation}
            \begin{aligned}
                \frac{\partial D}{\partial w} &= 2\,\text{softplus}(s_x - L_x) \cdot \frac{1}{1 + e^{-(s_x - L_x)}} \cdot \frac{ps_{x}}{\sqrt{ps_{x}^2 + \epsilon}} \cdot H_{point}(0,:) \cdot \begin{bmatrix} 1 \\ 0 \\ 0 \end{bmatrix}\\
                \frac{\partial D}{\partial d} &= 2\,\text{softplus}(s_y - L_y) \cdot \frac{1}{1 + e^{-(s_y - L_y)}} \cdot \frac{ps_{y}}{\sqrt{ps_{y}^2 + \epsilon}} \cdot H_{point}(1,:) \cdot \begin{bmatrix} 0 \\ 1 \\ 0 \end{bmatrix}\\
                \frac{\partial D}{\partial h} &= 2 \cdot ps_z \cdot H_{point}(2,:) \cdot \begin{bmatrix} 0 \\ 0 \\ 1 \end{bmatrix}
            \end{aligned}
        \end{equation}

    \section*{Derivatives with respect to the input point \(\mathbf{p}\)}

        The derivative of the squared distance with respect to the input point \(\mathbf{p}\) is given by:

        \begin{equation}
            \frac{\partial D}{\partial p} = \left[\frac{\partial D}{\partial p_{x}}, \frac{\partial D}{\partial p_{y}}, \frac{\partial D}{\partial p_{z}}\right]
        \end{equation}

        and $D$ is defined as:

        \begin{equation}
            D = ps_{z}^2 + dx^2 + dy^2
        \end{equation}

        The intermediate variables are defined as:

        \begin{align}
            dx^2 &= \left[\text{softplus}(\text{softabs}(ps_x) - L_x)\right]^2,\\
            u^2 &= [softplus(v)]^2 \\
            s &= \text{softabs}(ps_{x}) = \sqrt{ps_{x}^2 + \epsilon}, \\
            v &= s - L_x, \\
            ps_{x} &= H_{point}(0,:)\, p, \\
            L_x &= \frac{width}{2}.
        \end{align}

        \subsection*{Derivative of $D$ with respect to $p_x$}

            The derivative of \(D\) with respect to \(p_x\) is given by:
            \begin{equation}
                \frac{\partial D}{\partial p_x} = \frac{\partial D}{\partial dx^2} \cdot \frac{\partial dx^2}{\partial p_x}
            \end{equation}

            and since only the \(dx^2\) term depends on \(p_x\), we can compute the derivative of \(D\) with respect to \(dx^2\) first:

            \begin{equation}
                \frac{\partial D}{\partial p_x} = \frac{\partial dx^2}{\partial p_x}
            \end{equation}

            Apply the chain rule:
            
            \begin{equation}
                \frac{\partial dx^2}{\partial p_x}
                = \frac{\partial dx^2}{\partial u}
                \cdot \frac{\partial u}{\partial v}
                \cdot \frac{\partial v}{\partial s}
                \cdot \frac{\partial s}{\partial ps_{x}}
                \cdot \frac{\partial ps_{x}}{\partial p_x}
            \end{equation}
            
            We compute each term individually:

            \begin{align}
                \frac{\partial ps_{x}}{\partial p_x} &= H_{point}(0,0) \\
                \frac{\partial s}{\partial ps_{x}} &= \frac{ps_{x}}{\sqrt{ps_{x}^2 + \epsilon}}\\
                \frac{\partial v}{\partial s} &= 1\\
                \frac{\partial u}{\partial v} &= \frac{1}{1 + e^{-(s - L_x)}}\\
                \frac{\partial dx^2}{\partial u} &= 2\,\text{softplus}(s - L_x)\\
            \end{align}

           \subsubsection*{Combining all derivatives}
            
                Combining these terms, the complete derivative of \(dx\) with respect to \(p_x\) is:

                \begin{equation}
                    \frac{\partial dx^2}{\partial p_x}
                    = 2 \cdot \text{softplus}(s - L_x)
                    \cdot \frac{1}{1 + e^{-(s - L_x)}}
                    \cdot 1
                    \cdot \frac{ps_{x}}{\sqrt{ps_{x}^2 + \epsilon}}
                    \cdot H_{point}(0,0)
                \end{equation}

        \subsection*{Derivative of $D$ with respect to $p_y$}

            The derivative of \(D\) with respect to \(p_y\) is given by:
            \begin{equation}
                \frac{\partial D}{\partial p_y} = \frac{\partial D}{\partial dy^2} \cdot \frac{\partial dy^2}{\partial p_y}
            \end{equation}
            
            and since only the \(dy^2\) term depends on \(p_y\), we can compute the derivative of \(D\) with respect to \(dy^2\) first:
            
            \begin{equation}
                \frac{\partial D}{\partial p_y} = \frac{\partial dy^2}{\partial p_y}
            \end{equation}
            
            Apply the chain rule:
            
            \begin{equation}
                \frac{\partial dy^2}{\partial p_y}
                = \frac{\partial dy^2}{\partial u}
                \cdot \frac{\partial u}{\partial v}
                \cdot \frac{\partial v}{\partial s}
                \cdot \frac{\partial s}{\partial ps_{y}}
                \cdot \frac{\partial ps_{y}}{\partial p_y}
            \end{equation}
            
            We compute each term individually:
            
            \begin{align}
                \frac{\partial ps_{y}}{\partial p} &= H_{point}(1,1) \\
                \frac{\partial s}{\partial ps_{y}} &= \frac{ps_{y}}{\sqrt{ps_{y}^2 + \epsilon}}\\
                \frac{\partial v}{\partial s} &= 1\\
                \frac{\partial u}{\partial v} &= \frac{1}{1 + e^{-(s - L_y)}}\\
                \frac{\partial dy^2}{\partial u} &= 2\,\text{softplus}(s - L_{y}\\
            \end{align}
            
            
            \subsubsection*{Combining all derivatives}
            
            Combining these terms, the complete derivative of \(dy\) with respect to \(p_y\) is:
            
            \begin{equation}
                \frac{\partial dy^2}{\partial p_y} = 2 \cdot
                \text{softplus}(s - L_y)
                \cdot \frac{1}{1 + e^{-(s - L_y)}}
                \cdot 1
                \cdot \frac{ps_{y}}{\sqrt{ps_{y}^2 + \epsilon}}
                \cdot H_{point}(1,1).
            \end{equation}

        \subsection*{Derivative of $D$ with respect to $p_z$}

            The derivative of \(D\) with respect to \(p_z\) is given by:
            \begin{equation}
                \frac{\partial D}{\partial p_z} = \frac{\partial D}{\partial ps_{z}} \cdot \frac{\partial ps_{z}}{\partial p_z}
            \end{equation}

            and since only the \(ps_{z}\) term depends on \(p_z\), we can compute the derivative of \(D\) with the second factor only:

            \begin{equation}
                \frac{\partial D}{\partial p_z} = \frac{\partial ps_{z}}{\partial p_z} =  2\,ps_{z} \cdot H_{pose}(2,2)
            \end{equation}

        \subsection*{Final Jacobian of the distance with respect to \(p\)}
        The final Jacobian  is given by:

        \begin{equation}
            \begin{aligned}
                \frac{\partial D}{\partial p} &= \left[\frac{\partial D}{\partial p_{x}}, \frac{\partial D}{\partial p_{y}}, \frac{\partial D}{\partial p_{z}}\right] \\
                &= \Bigg[
                    2\,\text{softplus}(s_x - L_x) \cdot \frac{1}{1 + e^{-(s_x - L_x)}} \cdot \frac{ps_{x}}{\sqrt{ps_{x}^2 + \epsilon}} \cdot H_{\text{point}}(0,0), \\
                    &\quad\, 2\,\text{softplus}(s_y - L_y) \cdot \frac{1}{1 + e^{-(s_y - L_y)}} \cdot \frac{ps_{y}}{\sqrt{ps_{y}^2 + \epsilon}} \cdot H_{\text{point}}(1,1), \\
                    &\quad\, 2\,ps_{z} \cdot H_{\text{point}}(2,2)
                    \Bigg]
            \end{aligned}
        \end{equation}

\end{document}